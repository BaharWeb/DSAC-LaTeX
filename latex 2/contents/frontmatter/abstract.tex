% !TEX root = Clean-Thesis.tex
%
\pdfbookmark[0]{Abstract}{Abstract}
\chapter*{Abstract}
\label{chp:abstract}
%\vspace*{-10mm}

%\lipsum
Web Systems are widely used and accepted due to their advantages over traditional desktop applications.
Modernization of existing non-Web software towards the Web, however, is a complex and challenging task due to the characteristics of Legacy Systems.
Independent Software Vendors are struggling to commence Web Migration because of the involved effort and risk.
Through systematic field research and problem analysis, this situation is further analyzed, deriving a set of requirements that represent the effort and risk concerns and which are used to assess the state of the art in the field.
Existing Web Migration research exhibits gaps concerning dedicated approaches for the initial phase and feasibility of the proposed strategies with limited resources and expertise.

This thesis proposes a solution to address the shortcomings outlined above and to support Independent Software Vendors to commence Web Migration, focusing on their concerns about effort and risk.
The main idea is to provide a set of dedicated solutions to close the identified gaps in the form of a methodology and a supporting toolsuite that transfer paradigms successfully solving similar problems in other areas of computer science into the Web Migration domain.
These solutions constitute the proposed approach called Agile Web Migration for SMEs (AWSM), consisting of methods, tools, principles, and formalisms for reverse engineering, risk management, customer impact control, and migration strategy selection.
%\pagebreak

The thesis describes the research on the devised ideas in the context of a collaboration project with an Independent Software Vendor.
Applicability and feasibility of the concepts are
demonstrated in several evaluation experiments, integrating empirical user studies and objective measurements.
The thesis concludes with an evaluation based on requirements assessment and on the application of the solutions in the application scenario, and it provides an outlook towards future work.


%\vspace*{20mm}
%{\usekomafont{chapter}Abstract (different language)}\label{sec:abstract-diff} \\
%\blindtext

\cleardoublepage