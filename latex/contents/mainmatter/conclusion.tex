\hypertarget{sec:conclusion}{%
\chapter{Conclusion and Outlook}\label{sec:conclusion}}

To conclude the thesis, this provides a summary of the results and contributions as well as pointing out unresolved issues. Furthermore, this chapter highlights remaining research challenges.

\vspace{-15pt}
\hypertarget{thesis-summary}{%
\section{Thesis Summary}\label{thesis-summary}}
\vspace{15pt}
Motivated by the domain expert’s need for domain-specific situational decision-support tool in the time of overwhelming online data and complex technologies, the goal of this thesis was to develop a situational application to address this challenge. The proposed solution was designed to enable domain experts to integrate domain-specific components in the form of web composition leading to dedicated and cost-efficient solution.

The short lifespan of situational applications makes them unsuitable for conventional software development processes. Additionally, the overwhelming amount of data available online and the lack of sufficient technical skills among domain experts are key contributing challenges. These challenges can be categorized into three areas: poor usability, limited domain specificity, and insufficient expressive power. To address these challenges three research objectives have been identified: 1) developing an easy-to-use situational application for a specific domain 2) defining communication mechanisms to bridge the gap between domain expert's intentions and the final result 3) developing mechanisms for domain-specific customization of final tool. 

Two categories of requirements were identified based on the on the motivational scenario. Accordingly, a state-of-the-art analysis was conducted to investigate existing solutions and identify their shortcomings. A thorough analysis revealed that current approaches fail to fully address the unique characteristics of domain-specific situational decision-support tools.
The proposed solution is called \gls{dsac}, emphasized domain specificity by considering the limited technical expertise of domain experts. To implement this solution five tools were developed, each addressing a specific subproblem. These tools are grounded in three primary mechanisms: Model-Driven Composition Development, Conversational-Based Discovery and Composition, and Ontology-Based Platform Development. To ensure the overall quality of the development method, a set of design principles was established. These principles priorities reusability,  domain specificity, user-centric design, gentle learning slop, and responsibility segregation. At the core of the solution is the Model-driven Composition Platform, which provides an \gls{api} composition environment capable of generating the platform-compliant components and executable compositions. To address the poor expressiveness, a chatbot for component discovery and composition was introduced through the DisCo and Solution Designer tools. As previously mentioned, the primary focus of this thesis is domain specificity, addressed by the Domain Analyzer. This tool generates the necessary domain models and configures the final tool based on the specific characteristics of the target domain. 

Each tool is evaluated individually according to the objectives it is meant to achieve. The evaluation process included user studies and data-driven analysis. Participants rated the composition tool as easy to use and effective; however, some reported issues regarding the component \gls{ui} generation.  Overall, the analysis indicates that the platform successfully met the identified objectives. 


\hypertarget{lessons-learned}{%
\section{Lessons Learned}\label{lessons-learned}}
\vspace{15pt}

The research conducted in the scope of this thesis revealed interesting findings that are not directly related to the \gls{dsac} toolkit but have importance for research in the field of \gls{eud} in general: 

\begin{itemize}
\tightlist
\item
  Domain experts are capable of performing composition related tasks and UI navigations. 
\item
  Integrating chatbots and dialogue-based interfaces with visual
  guidance can increase end user's engagement and confidence.
\item
  Full automated workflow decreases end user's confidence and can be a
  confusion cause. End users tend to be involved in development process
  to alight the final result based on their needs.
\item
  Large language models (LLMs) alone often generate data that can be
  unstructured or chaotic. it is advisable to combine LLMs with
  complementary techniques.
\item
  Users tend to exhibit consistent behavioral patterns within their
  familiar environment, such as in mobile application or specific
  operating systems.
\item
  Even domain experts may face challenges in expressing their key
  requirements and formulating the right questions.
\end{itemize}


\hypertarget{sec:conclusion.contributions}{%
\section{Summary of Contributions}\label{sec:conclusion.contributions}}
\vspace{15pt}

The research presented in this thesis led to the development of toolkits including algorithms, models, and architectures to support development of domain-specific \gls{eud} tools for situational decision-making. These applications are aligned with the domain expert’s technical skills. The successful achievement of this goal has resulted in the research contributions detailed in \cref{sec:introduction.contributions}. Following is a detailed summary of these contributions: 

\begin{itemize}
\item
  Definition of models, algorithms, and architecture of a situational
  decision-support tool align with domain-specific use cases.
\item
  Development of a component model compliant with OpenAPI specification
  structure through a model-to-model transformation.
\item
  Definition of a conversion mechanism to generate full-fledged
  components from existing \gls{api}s including the UI and execution code
  through a model-to-code transformation.
\item
  Development of model and architecture of a technology-independent
  composition model integrated with domain business rules.
\item
  Development of a \gls{api} composition design mechanism to identify the
  compatible components and generate a graph-based composition sequence.
\item
  Development of a discovery chatbot for identifying and retrieving \gls{api}s
  according to domain expert's query.
\item
  Specification of end-user-friendly interaction mechanism and
  visualization assistance elements.
\item
  Development of algorithms and tools to generate domain ontology using
  LLMs with domain expert involvement.
\item
  Development of a mechanism to generate domain business rules based on
  domain ontology in the form of NL-based rules and composition
  restrictions.
\item
  Specification of component's semantic annotations and mapping based on
  the domain ontology.
\end{itemize}


\vspace{-15pt}
\hypertarget{ongoing-and-future-work}{%
\section{Ongoing and Future Work}\label{ongoing-and-future-work}}
\vspace{15pt}

The \gls{dsac} platform serves as a robust solution for addressing the challenges faced by domain experts in situational decision-making, with a dedicated focus on domain specificity. The proposed platform designed to enhance usability and expressiveness while abstracting the complexities of underlying technologies. The Component Generator and \gls{dsac} Composer offer an intuitive composition environment. Disco supports an efficient and interactive discovery mechanism and Domain Analyzer plays a pivotal role in generating the domain models required for domain-specific customization. 

\hypertarget{tools-improvements}{%
\subsection{Tools and Mechanisms Improvement}\label{tools-improvements}}
\vspace{10pt}

The toolkit proposed in this thesis can be improved in terms of efficiency and usability. Throughout the rest of this section potential improvements for each tool is presented.

\textbf{Component Generator} tool generates platform-compliant
components from existing \gls{api}s based on their OpenAPI specification.
Currently, the tool supports OpenAPI v3 documents. However, many APIs
still use earlier versions for documentation. A potential direction for
future work is to incorporate a conversion extension or develop
functionality to generate OpenAPI specifications for \gls{api}s that lack
them. This enhancement enables the tool to take advantage of a wider
range of \gls{api}s available online.

\textbf{\gls{dsac} Composer} produces interactive dashboards allowing users to
aggregate components to generate a uniform solution within a single
interface. However, managing component on the screen could become
challenging in the case of complex solutions. to address this, a proper
component grouping mechanism could enhance the usability. As another
possible extension, the current platform can benefit from the
multitenant architecture, enabling multiple users to collaborate and be
supported simultaneously. This would enhance scalability and
adaptability, making the platform more robust in dynamic scenarios.

\textbf{DisCo} relies on ProgrammableWeb as its primary source for \gls{api}
mining; however, this platform was discontinued in 2023. Future work
should focus on making the \gls{api} repository generation
platform-independent, enabling extraction of the textual description and
necessary annotations directly from \gls{api} documentation. Furthermore,
analyzing the patterns in user queries could provide more personalized
responses. Adding features like auto-complete or query suggestions could
further increase the expressiveness and usability.

\textbf{Solution Designer} generates solution classes with manageable
complexity; optimizing the algorithm\textquotesingle s performance is
essential to ensure acceptable response times, for more complex queries
and large-scale component sets. Future work could also explore
incorporating the user's behavior patterns and context-aware filtering
for to enhance query handling and improve overall system efficiency.

\textbf{Domain Analyzer} tool facilitates the generation of domain
models and applies the domain-specific configurations. Potential
improvements could focus on enhancing the quality of the domain models
and the extracted rules, particularly in complex domain scenario. The
business rules used in this work, are in natural language and RDF
formats; these should be further developed to enable conversion into
formal representations, such as SQL, for integration as database rules.
Moreover, incorporating dynamic domain rules (at the UI level) based on
user interactions into the DOM node could enhance the tools usability
and interactivity.

\vspace{-10pt}
\hypertarget{open-questions}{%
\subsection{Open Questions}\label{open-questions}}
\vspace{10pt}

Future work should tackle the following open research questions.

The first open question concerns the development of two assistance
mechanisms to further engage domain experts in the development phases.
As discussed in \cref{sec:dsac-composition-platform}, the tool automates the component generation
process, including UI and client code generation. This decision was
driven by the process complexity and considerations regarding the
limited technical skills of domain experts. However, if any component
generation step fails -- for instance if the generated UI is inaccurate
due to compatibility issues or errors in the OpenAPI specification
document, or if the client code is non-executable -- domain experts are
unable to proceed with using the solution or intervene to resolve the
problem. Therefore, the component developers should address the problem
and provide the appropriate support. To alleviate this situation, an
assistance mechanism can delegate some of these tasks to domain experts
by offering step-by-step guidance. Another assistance mechanism can be
facilitating domain model generation. Although domain experts are
already involved in this process, the model extension still requires the
developer's supervision.

Another potential future direction for this work is the extension of the
framework to support situational decision-making in multidisciplinary
environments. The popularity of multidisciplinary projects has been on
the rise, motivated by complex challenges that single disciplines cannot
tackle alone. Incorporating support for such contexts would
significantly enhance the tool's utility and applicability. This
extension would require several key steps including, mapping multiple
domain ontologies using alignment techniques to expand their semantic
coverage, addressing multi-modal \gls{api} compatibility issues, and
incorporating AI-driven orchestration mechanisms. This would make the
border adaptation to complex scenarios, particularly in projects that
demand collaboration across diverse domains of expertise.

Device applicability is another topic that requires further
investigation. Although the \gls{dsac} platform is mainly designed as a
desktop application, developing a standalone mobile app would be a
potential future work direction. This requires addressing challenges
related to platform compatibility, user interface adaptation for smaller
screens, efficient resource management on mobile devices, and impending
other input methods such as speech interaction.
