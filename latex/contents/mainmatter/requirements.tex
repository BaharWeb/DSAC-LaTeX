\hypertarget{sec:requirements-analysis}{%
\chapter{Requirements Analysis}\label{sec:requirements-analysis}}

This chapter provides a more detailed overview of the central problem raised in the introduction, regarding the insufficient support for developing domain-specific situational applications. To offer better guidance, a use case scenario is presented in Section 2.1.1. This scenario helps to systematically identify key stakeholders and analyze central problem. Finally, Section 2.2, outlines the accurate requirements to address the identified problem.

\vspace{-15pt}
\hypertarget{sec:problem-analysis}{%
\section{Problem Analysis}\label{sec:problem-analysis}}
\vspace{15pt}

This section outlines a systematic approach to identifying the root causes and effects of the central problem. We follow the use case methodology in software engineering to identify, analysis and clarify the system requirements. Utilizing scenarios in problem analysis phase enhances our understanding of involved stakeholders and application requirements \autocite{Dzida1998}. Subsection 2.1.1 describes a situation in which an envisioned domain-specific situational application is used to help domain experts achieving their goals. The involved stakeholders in this scenario are characterized in 2.1.2.

\vspace{-10pt}
\hypertarget{sec:scenario}{%
\subsection{Scenario}\label{sec:scenario}}
\vspace{10pt}

This section provides a detailed scenario to establish the context and highlight the importance of a domain-specific situational decision-support tool. The scenario’s narrative highlights the issues of poor expressiveness, limited usability, and limited domain specificity of current situational applications. Alice is working in a research group, and she is tasked with organizing a small educational event aimed at exchanging knowledge on the latest publications, articles, and trends in Web Engineering. For this aim Alice should make informed decisions such as choosing trending topics, identifying, and selecting keynote speakers based on their publication’s significance and finally making decisions regarding event planning such as venue selection and agenda preparation. 
To select potential speakers, Alice should navigate research databases and search for the state-of-the-art publications based on criteria such as research impact, alignment with her research group’s focus area, and current trends in Web Engineering. Alice decided to collect information using scholarly APIs such as IEEE Xplore API\footnote{\url{https://developer.ieee.org/}} or Microsoft Academic Knowledge API\footnote{\url{https://www.microsoft.com/en-us/research/project/academic/articles/sign-academic-knowledge-api/}}. This data should be then consumed by other APIs for analysis and reporting purposes. This step helps Alice to pinpoint the research groups with the highest impact in a particular topic and figuring out which author has publications closely aligned with her research focus. Manually going through this large volume of research publications can be overwhelming, potentially causing information overload and increasing the risk of missing important content. 
Alice requires a solution to support her with  these immediate needs. Such solutions have a short lifespan and are not applicable for Alice after the event. In such cases, traditional software development processes are not cost and time efficient.
Moreover, upon a thorough scenario examination, it becomes evident that it is a domain-specific situation relying entirely on typical concepts of the research and academic domain. Domain concepts such as research publication, h-index, scholarly database, and citation rate are familiar only to domain experts. Therefore, generic mashup tools are not sufficient to address domain-specific requirements in this scenario. On the other hand, using natural language as the medium for user’s interactions will increase the usability and reduce the gap between user’s intention and final solution \autocite{Zarei2021}. With this in mind, Alice is encouraged to simply type in her requirements and communicate with the tool to discover essential APIs and compose them if necessary.


\vspace{-10pt}
\hypertarget{sec:stakeholder-analysis}{%
\subsection{Stakeholder Analysis }\label{sec:stakeholder-analysis}}
\vspace{10pt}

Stakeholders are individuals, organizations or groups that impact or being impacted by the solution result. Identifying the main stakeholders and understanding their contribution to the solution’s success or failure is a crucial step \autocite{Brugha2000}. The identified stakeholders in this thesis are outlined below. A detailed table of stockholder analysis is presented in the Appendix.

\textbf{Individuals}. This group comprises domain experts (also known as
knowledge workers) and web developers. Domain experts possess
domain-specific expertise in a particular field, such as medicine,
accounting, or business \autocite{Chemnitz2017}. They have enough
skills to use the internet for their situational purposes but lack
expertise in web or software programming. In contrast, web developers
are informatics-educated users who may not be familiar with
domain-specific concepts, operations, or rules. The communication
between domain experts and web developers is pivotal to minimize
ambiguity and increasing the tool's usability and precision. Individuals
(domain expert or web developer) face a number of challenges regarding
the usability and flexibility of the existing tools. The focus of this
work is on domain experts therefore, we consider domain experts as end
users and use these two terms interchangeability.

\textbf{Service Providers}. This group comprises organizations providing
and delivering services to users such as developers, IT companies,
online shopping, etc. \autocite{Chemnitz2017}. Service providers
design APIs as seamless communication medium between various services,
enabling interoperability and data exchange. They play a crucial role as
stakeholder in the software development and deployment process. Although
service providers often lack in depth domain knowledge to fully
comprehend domain expert's requirements.

\vspace{-15pt}
\hypertarget{sec:requirements}{%
\section{Requirement Analysis}\label{sec:requirements}}
\vspace{15pt}

Inspired by the presented scenario, objective statements, and stakeholder characteristics, requirements for developing a domain-specific situational application for decision-making purposes are elicited and organized into two groups. The first group focuses on the development process, while the second covers tool-related requirements from the end user's perspective. These requirements serve as a guideline for conducting the literature review and systematic assessment in Chapter 3.
The success of this thesis depends on the fulfillment of the specified requirements. To assess the extent to which each requirement is fulfilled, an assessment scheme is introduced with three levels: fully satisfied, partially satisfied, and not satisfied. For the sake of clarity, a consistent naming convention is adopted to reference each requirement throughout the thesis. \cref{tbl:requirements} summarizes requirements and corresponding objectives addressed by each.

\hypertarget{tbl:requirements}{}
\begin{longtable}{@{}lll@{}}
\caption{\label{tbl:requirements}PR and TR Requirements with Related Objectives}\tabularnewline
\toprule
\begin{minipage}[b]{0.18\columnwidth}\raggedright
Requirement ID\strut
\end{minipage} & 
\begin{minipage}[b]{0.45\columnwidth}\raggedright
Description\strut
\end{minipage} & 
\begin{minipage}[b]{0.27\columnwidth}\raggedright
Related Objectives\strut
\end{minipage}\tabularnewline
\midrule
\endfirsthead

\toprule
\begin{minipage}[b]{0.18\columnwidth}\raggedright
Requirement ID\strut
\end{minipage} & 
\begin{minipage}[b]{0.45\columnwidth}\raggedright
Description\strut
\end{minipage} & 
\begin{minipage}[b]{0.27\columnwidth}\raggedright
Related Objectives\strut
\end{minipage}\tabularnewline
\midrule
\endhead

\begin{minipage}[t]{0.18\columnwidth}\raggedright
PR1\strut
\end{minipage} & 
\begin{minipage}[t]{0.45\columnwidth}\raggedright
Semi-Automation\strut
\end{minipage} & 
\begin{minipage}[t]{0.27\columnwidth}\raggedright
\cref{ro:1}, \cref{ro:2}, \cref{ro:3}\strut
\end{minipage}\tabularnewline

\begin{minipage}[t]{0.18\columnwidth}\raggedright
PR2\strut
\end{minipage} & 
\begin{minipage}[t]{0.45\columnwidth}\raggedright
Systematic Reuse\strut
\end{minipage} & 
\begin{minipage}[t]{0.27\columnwidth}\raggedright
\cref{ro:1}\strut
\end{minipage}\tabularnewline

\begin{minipage}[t]{0.18\columnwidth}\raggedright
PR3\strut
\end{minipage} & 
\begin{minipage}[t]{0.45\columnwidth}\raggedright
Functionality Integration\strut
\end{minipage} & 
\begin{minipage}[t]{0.27\columnwidth}\raggedright
\cref{ro:1}, \cref{ro:2}\strut
\end{minipage}\tabularnewline

\begin{minipage}[t]{0.18\columnwidth}\raggedright
TR1\strut
\end{minipage} & 
\begin{minipage}[t]{0.45\columnwidth}\raggedright
Interactive User Interface\strut
\end{minipage} & 
\begin{minipage}[t]{0.27\columnwidth}\raggedright
\cref{ro:2}\strut
\end{minipage}\tabularnewline

\begin{minipage}[t]{0.18\columnwidth}\raggedright
TR2\strut
\end{minipage} & 
\begin{minipage}[t]{0.45\columnwidth}\raggedright
Domain Specificity\strut
\end{minipage} & 
\begin{minipage}[t]{0.27\columnwidth}\raggedright
\cref{ro:3}\strut
\end{minipage}\tabularnewline

\begin{minipage}[t]{0.18\columnwidth}\raggedright
TR3\strut
\end{minipage} & 
\begin{minipage}[t]{0.45\columnwidth}\raggedright
Usability\strut
\end{minipage} & 
\begin{minipage}[t]{0.27\columnwidth}\raggedright
\cref{ro:1}, \cref{ro:3}\strut
\end{minipage}\tabularnewline

\begin{minipage}[t]{0.18\columnwidth}\raggedright
TR4\strut
\end{minipage} & 
\begin{minipage}[t]{0.45\columnwidth}\raggedright
Effectiveness\strut
\end{minipage} & 
\begin{minipage}[t]{0.27\columnwidth}\raggedright
\cref{ro:1}, \cref{ro:2}, \cref{ro:3}\strut
\end{minipage}\tabularnewline

\bottomrule
\end{longtable}

\vspace{-10pt}
\hypertarget{sec:dp-requirements}{%
\subsection{Development Process Requirement}\label{sec:dp-requirements}}
\vspace{10pt}

The scenario outlined in Section 2.1.1 highlighted the demand for a domain-specific situational tool to facilitate the decision-making process. In this context, the development process should involve domain experts to maximize the domain knowledge utilization. Additionally, the process should enable the integration of existing functionalities to ensure efficiency in terms of time and effort. These considerations lead us to the subsequent development process requirements.

\begin{thesisdprequirement}{Semi-Automation}{pr:1}
Automation is defined as the degree to which the tool performs the development process on behalf of domain experts. The levels of automation range from full automation to semi-automation or manual. In fully automated solutions, the tool takes complete control over the development process, with end users primarily contributing inputs and validating the final results. In such tools, The learning overload for domain experts is relatively low although the result can be imprecise since the automatically produced tool is not directly deviated from the domain expert's needs \autocite{Aghaee2014}. On the other hand, manual solutions expose users to an extensive amount of programming and development activities. From development perspective, manual tools are relatively easy to develop. Nevertheless, from the end user’s point of view, the complexity is relatively high. As discussed in Section 1.2 and according to \autocite{Blili1998} domain experts' involvement can bring several advantages from operational and economical aspects. Therefore, a semi-automated tool is desirable in this context. Such a tool supports domain experts by providing guidance throughout the various phases of the development process. Domain experts have the flexibility to directly intervene and modify the tool's building blocks to align them with their specific requirements without impairing the security or reliability.

This requirement is fully satisfied in the case of semi-automated solutions, which allow direct involvement of domain experts and technical users while providing the required guidance throughout development, maintenance, and solution extension. It is partially satisfied in the case of fully automated solutions and not satisfied for manual solutions. 

\end{thesisdprequirement}

%(Target Environment)
\begin{thesisdprequirement}{Systematic Reuse}{pr:2}
Reusability is a vital requirement for both web and software applications. Enforcing systematic reuse can guarantee efficiency in terms of cost and time and overall product quality \autocite{Schmidt1999a}. The development process of situational application should promote creation, discovery, and reuse of software artifacts. 
This criterion is fully satisfied if the development process actively encourages software artifacts reuse and creation. It is partially satisfied if the development process concentrates solely on either artifact creation or reuse and not satisfied if the artifact's reuse is not enforced during the development process.

\end{thesisdprequirement}

\begin{thesisdprequirement}{Functionality Integration}{pr:3}
As outlined in the scenario, Alice requires a composite solution to orchestrate multiple web APIs and provide a unified view of the generated data, leading to a class of methods known as integrated solutions. In dynamic environments, domain experts are constantly facing emerging tasks; therefore, the need for service integration is inevitable \autocite{Daniel2014a}. The envisioned tool should facilitate data and service integration in an easy and affordable manner. 
This requirement is fully satisfied if the tool allows data and functionality integration and provides necessary guidance to accomplish this task. Partially satisfying this requirement means the tool supports integration but lacks efficiency or simplicity, resulting in integration failure. If the tool does not provide any possibility for functionality integration, then this criterion is considered not satisfied.

\end{thesisdprequirement}


\subsection{Tool Requirements}\label{sec:tool-requirements}
\vspace{10pt}
Aligned with the presented scenario and stated objectives, the tool should provide a domain-specific situational solution considering usability, expressiveness, and domain specificity. To systematically define the criteria for the envisioned tool, the Tool Requirements category is introduced , focusing the tool’s characteristics from the user's perspective.

\begin{thesistoolrequirement}{Interactive User Interface}{tr:1}
As discussed in Section 1.3, the usability of a software tool is heavily influenced by the interaction between the domain expert and the tool. According to \autocite{Fuckner2013}, providing domain experts with natural language-based interaction mechanisms can increase efficiency and usability. Moreover, it shapes domain experts’ behavior in a further informative manner resulting in more precise result \autocite{Zarei2020}. Leveraging natural language as the interaction medium lifts the learning burden and reduce the complexity posed on domain experts. Having these considerations in mind, it is essential for the tool to have interactive user interface to address the usability and expressiveness objectives (\cref{ro:1} and \cref{ro:2} respectively).
The criterion is considered fully satisfied if the tool integrates interactive user interface and allows domain experts to communicate in natural language. The requirement is partially satisfied if the tool offers interaction mechanisms that uses constrained natural languages, which impose an additional learning burden on users. The requirement considered not satisfied if the tool failed to provide a interactive user interfaces and interaction means.

\end{thesistoolrequirement}

\begin{thesistoolrequirement}{Domain Specificity}{tr:2}
To effectively support domain experts during the decision-making process, the tool should be tailored to domain-specific requirements and knowledge \autocite{Soi2010}. From \gls{eud} and \gls{hci} perspective, domain specificity is a fundamental quality guaranteeing the end result accuracy. To achieve this goal it is crucial to confine the tool within a well-defined domain and configure it to align with domain-specific concepts, operations, and rules \autocite{Desolda2017}.
This requirement is fully satisfied if the tool is domain-specific in terms of terminology, operation, components, and business rules. It is partially satisfied, if the tool failed to provide domain specificity in any of above-mentioned aspects. The criterion is not satisfied if the end result is a general prepose tool failing to provide domain experts with familiar situational application.

\end{thesistoolrequirement}

\begin{thesistoolrequirement}{Usability}{tr:3}
As discussed in Section 1.3 and concerning research objective \cref{ro:1}, addressing usability for situational decision-support tools has significant importance that manifests itself in tool acceptance and domain expert’s satisfaction. Usability covers a wide range of metrics such as readability, familiarity, or predictability  \autocite{Insfran2012}. One of the primary visions within the HCI is to address stakeholder’s goals without imposing an excessive burden of time and effort related to understanding or managing the underlying technology  \autocite{Ponce2022}.
Within the scope of this work the tool’s usability is evaluated by the imposed learning curve, language and interaction patterns and terminology familiarity.
This requirement is fully satisfied if the tool is easy to use while using familiar terminology. It is partially satisfied if the tool lacks domain familiarity in terms of terminology and interaction patterns nonetheless the learning curve remains relatively moderate. Finally, it is not satisfied if the imposed challenges on the domain experts is high and complex.

\end{thesistoolrequirement}

\begin{thesistoolrequirement}{Effectiveness}{tr:4}
The last requirement in this category concerns the tool's effectiveness in supporting domain experts at a designated level of quality. In software engineering, effectiveness is defined as the extent to which a tool can successfully achieve the end user’s objectives and deliver values. Higher effectiveness ensures higher productivity and impactful experience for the domain experts. Effectiveness is measured though metrics such as quality, usability, and application completeness which is determined as the level of application completion accordance with the user's requirements \autocite{Semiawan2021}. The application quality and usability are assessed via user’s test.
The effectiveness requirement is fully satisfied if the tool produces a complete solution with an accepted level of usability and quality, as determined by user tests and questionnaires. It is partially satisfied if the tool only partially satisfies the metrics for completeness, quality, and usability. Finally, the requirement is not met if the tool fails to provide a complete result and/or the result does not address the requirements at all and/or has inadequate usability.

\end{thesistoolrequirement}

\vspace{-15pt}
\hypertarget{sec:requirements.summary}{%
\section{Summary}\label{sec:requirements.summary}}
\vspace{15pt}

This chapter provides an in-depth analysis of the requirements for realizing a solution that addresses the problems outlined in Chapter 1. To achieve this, a guiding scenario is presented to better highlight the struggles of domain experts in the context of situational decision-making. Scenario analysis highlighted problems related to limited usability, poor expressiveness, and limited domain specificity. Moreover, two groups of stockholders were identified as individuals and service providers. Aligned with the identified problem, scenario, and thesis objectives, seven requirements were elicited and categorized into Process Requirements and Tool Requirements. The process requirements focus on development-related aspects such as semi-automation, systematic reusability, and functionality integration. The tool requirements concentrate on the necessary characteristics of a domain-specific situational tool from domain expert’s perspective including Interactive User Interface, Domain Specificity, Usability, and Effectiveness. Each requirement is assessed based on a three-level assessment scheme, which is also applied in the next chapter to analyze existing solutions. 