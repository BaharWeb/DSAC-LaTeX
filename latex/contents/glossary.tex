\newglossaryentry{artifact} {
  name=Artifact,
  text=artifact,
  plural=artifacts,
  first={\emph{software artifact}}, 
  firstplural={\emph{software artifacts}}, 
  description={``A software artifact is a tangible machine-readable document created during software development. Examples are requirement specification documents, design documents, source code, and executables.'' \autocite{OMG2016KDM}, \autocite[cf.~\emph{physical asset}][entry 3.261]{ISO/IEEE24765Vocabulary}}
}
\newglossaryentry{asset} {
  name=Asset,
  text=asset,
  plural=assets,
  first={\emph{software asset}}, 
  firstplural={\emph{software assets}}, 
  description={An ``item, thing or entity that has potential or actual value to an organization'' \autocite{ISO/IEEE24765Vocabulary}. ``A software asset is a description of a partial solution \ldots{} or knowledge \ldots{} that engineers use to build or modify software products.'' \autocite{OMG2016KDM}, \autocite[cf.~\emph{intangible assets}][entry 3.261]{ISO/IEEE24765Vocabulary}}
}
\newglossaryentry{business case} {
  name=Business case,
  text=business case,
  plural=business cases,
  first={\emph{business case}},
  description={A ``documented economic feasibility study used to establish validity of the benefits of a selected component lacking sufficient definition and that is used as a basis for the authorization of further project management activities'' \autocite[][entry 3.442]{ISO/IEEE24765Vocabulary}}
}
\newglossaryentry{Concept} {
  name=Concept,
  description={``Concepts are units of human knowledge that can be processed by the human mind (short-term memory) in one instance.'' \autocite{Rajlich2002Concepts}, defined in \Cref{def:concept}}
}
\newglossaryentry{Concept Assignment} {
  name=Concept Assignment,
  description={A reverse engineering technique that aims at discovering human-oriented \glspl{Concept} and assigning them to their realizations in the source code \autocite{Biggerstaff1994ConceptAssignmentJournal}, cf. \cref{sec:re.related}}
}
\newglossaryentry{Crowdsourcing} {
  name=Crowdsourcing,
  description={``The act of a company or institution taking a function once performed by employees and outsourcing it to an undefined (and generally large) network of people in the form of an open call.'' \autocite{Howe2006}, defined in \Cref{def:crowdsourcing}}
}
\newglossaryentry{Desktop Application} {
  name=Desktop Application,
  description={Application software which has a \gls{gui}, or, less commonly, a \gls{tui}, that is not based on \web technologies in contrast to \glspl{Web Application}. A comparison can be found in \cref{sec:situation}}
}
\newglossaryentry{Encapsulation} {
  name=Encapsulation,
  description={Encapsulation approaches are \gls{Web Migration} approaches, that wrap the unchanged \gls{Legacy System} or parts of it and expose a new interface which is then integrated with the target system. Defined in \cref{sec:approaches}}
}
\newglossaryentry{Forward Engineering} {
  name=Forward Engineering,
  text=Forward Engineering,
  first={Forward Engineering}, 
  description={The application of Software Engineering to create a new software based on software requirements or on the results of \gls{Reverse Engineering}, when part of \gls{Reengineering}.}
}
\newglossaryentry{Legacy Modernization} {
  name=Legacy Modernization,
  text=legacy modernization,
  see={Software Modernization},
  description={see \gls{Software Modernization}}
}
\newglossaryentry{Hybrid Web Application} {
  name=Hybrid Web Application,
  description={A \gls{Web Application} that consists of parts of reused legacy code, a transformed \web UI and a generated RESTful API, cf. \cref{sec:rwmp}.}
}
\newglossaryentry{Legacy System} {
  name=Legacy System,
  description={``Any systems that cannot be modified to adapt to continually changing business requirements and their failure can have a severe impact on business.'' \autocite{Brodie1995Migrating}, defined in \Cref{def:legacysystem}}
}
\newglossaryentry{metamodel} {
  name=Metamodel,
  text=metamodel,
  first={\emph{metamodel}}, 
  description={A ``logical information model that specifies the modeling elements used within another (or the same) modeling notation'' \autocite[][entry 3.2433]{ISO/IEEE24765Vocabulary}}
}
\newglossaryentry{migrationengineer} {
  name=Migration Engineer,
  text=Migration Engineer,
  first={\emph{Migration Engineer}}, 
  description={A Software Engineer who is performing \gls{Web Migration} activities. Performing \gls{Web Migration} activities does not necessarily imply \gls{Web Migration} expertise. Migration Engineers are to be considered a sub-class of Software Engineers, i.e.~all characteristics of Software Engineers apply.}
}
\newglossaryentry{Prototyping} {
  name=Prototyping,
  see={Software Prototyping},
  description={}
}
\newglossaryentry{Rapid Prototyping} {
  name=Rapid Prototyping,
  description={A ``type of prototyping in which emphasis is placed on developing prototypes early in the development process to permit early feedback and analysis in support of the development process'' \autocite{ISO/IEEE24765Vocabulary}, defined in \Cref{def:rapidprototyping}}
}
\newglossaryentry{Rapid Web Migration Prototyping} {
  name=Rapid Web Migration Prototyping,
  first={\emph{Rapid Web Migration Prototyping}},
  description={A type of \gls{Rapid Prototyping} which allows the rapid creation of demonstrative, horizontal prototypes, that represent future \glslink{web}{Web}-based versions of existing non-\gls{web} \gls{Legacy System} allowing to assess and demonstrate their plausibility and desirability for decision making in \gls{Web Migration}. Defined in \cref{def:rwmp}}
}
\newglossaryentry{Reengineering} {
  name=Reengineering,
  text=Reengineering,
  first={Reengineering}, 
  description={The ``examination and alteration of software to reconstitute it in a new form, including the subsequent implementation of the new form'' \autocite[][entry 3.3346]{ISO/IEEE24765Vocabulary}, it consists of \gls{Reverse Engineering} and \gls{Forward Engineering}}
}
\newglossaryentry{Reverse Engineering} {
  name=Reverse Engineering,
  text=Reverse Engineering,
  first={Reverse Engineering}, 
  description={A ``software engineering approach that derives a system's design or requirements from its code'' \autocite[][entry 3.3501]{ISO/IEEE24765Vocabulary}}
}
\newglossaryentry{risk management} {
  name=Risk Management,
  text=risk management,
  first={\emph{risk management}}, 
  description={An ``organized process for identifying and handling risk factors'' \autocite[][entry 3.3528]{ISO/IEEE24765Vocabulary}}
}
\newglossaryentry{Software Migration} {
  name=Software Migration,
  plural=Software Migrations,
  first={\emph{Software Migration}}, 
  description={modification of software to run in different environments, a software maintenance technique and a part of a software's lifecycle \autocite[][p. 5-10]{SWEBOK2014}}
}
\newglossaryentry{Software Modernization} {
  name=Software Modernization,
  first={\emph{Software Modernization}}, 
  description={The ``process of evolving existing software systems by replacing, re-developing, reusing, or migrating the software components and platforms, when traditional maintenance practices can no longer achieve the desired system properties'' \autocite[][6]{Khadka2016PHD}}
}
\newglossaryentry{Software Prototyping} {
  name=Software Prototyping, 
  description={``Software prototyping is an activity that generally creates incomplete or minimally functional versions of a software application, usually for trying out specific new features, soliciting feedback on software requirements or user interfaces, further exploring software requirements, software design, or implementation options, and/or gaining some other useful insight into the software.'' \autocite{SWEBOK2014}, defined in \Cref{def:prototyping}}
}
\glsadd{Software Prototyping}
\newglossaryentry{source system} {
  name=Source System,
  text=source system,
  plural=source systems,
  first={source system}, 
  description={The existing software system from which the \gls{target system} is created during \gls{Software Migration}. Typically, source systems are \glspl{Legacy System}.}
}
\newglossaryentry{target environment} {
  name=Target Environment,
  text=target environment,
  plural=target environments,
  first={\emph{target environment}}, 
  description={The specific environment for which a software is modified in \gls{Software Migration}}
}
\newglossaryentry{target system} {
  name=Target System,
  text=target system,
  plural=target systems,
  first={target system}, 
  description={The software system to be created through \gls{Software Migration} in the intended \gls{target environment}, based on an existing \gls{source system}. For \gls{Web Migration}, target systems are \glspl{Web System}.}
}
\newglossaryentry{Technical Debt} {
  name=Technical Debt,
  description={``Technical debt is a collection of design or implementation constructs that are expedient in the short term but set up a technical context that can make future changes more costly or impossible.'' \autocite{Avgeriou2016TD}}
}
\newglossaryentry{Transformation} {
  name=Transformation,
  description={Transformation approaches are \gls{Web Migration} approaches that process the \glslink{Legacy System}{legacy} source code by a series of automatic steps turning it into the target system's source code. Defined in \cref{sec:approaches}}.
}
\newglossaryentry{Web Application} {
  name=Web Application,
  first={\emph{Web Application}},
  description={``A Web Application is a software system based on technologies and standards of the World Wide Web Consortium (W3C) that provides Web specific resources such as content and services through a user interface, the Web browser.'' \autocite{Kappel2006WebEngineering}, defined in \Cref{def:webapplication}}
}
\newglossaryentry{Web Engineering} {
  name=Web Engineering,
  first={\emph{Web Engineering}},
  description={``Web Engineering is the application of systematic, disciplined and quantifiable approaches to development, operation, and maintenance of Web-based applications.'' \autocite{Deshpande2002WebEngineering}}
}
\newglossaryentry{Web System} {
  name=Web System,
  description={A Web System is a software system based on technologies and standards of the \gls{w3c} that provides Web specific resources. \autocite[adapted from][]{Gaedke2000Diss,Kappel2006WebEngineering}, defined in \Cref{def:websystem}}
}
\newglossaryentry{Web Systems Evolution} {
  name=Web Systems Evolution,
  description={An area of \gls{Software Modernization} research, where both the target system and also the source system is a \gls{Web System}. \autocite[cf.][]{Kienle2014EvolutionWeb}}
}
\newglossaryentry{Web Migration} {
  name=Web Migration,
  description={Web Migration is a type of \gls{Software Migration} which transfers a \emph{non-\web source system} to a \emph{\web-based target environment}. It is \emph{adaptive and perfective maintenance} adapting software systems for web-based environments and improving functionality and maintainability. \autocite{ISO/IEEE2006SoftwareLifeCycle}}
}
\newglossaryentry{web migration prototype} {
  name=Web Migration Prototype,
  text=web migration prototype,
  description={A limited \web-based version of the \gls{Legacy System} which serves as \emph{functional, demonstrative product} \autocite{ISO/IEEE24765Vocabulary}, showing the plausibility and desirability. Web migration prototypes are \emph{demonstrative prototypes} which communicate the ideas of the user interface and capabilities of the envisioned \gls{Web System}, allowing decision makers to assess desirability and benefits \autocite{Wallmueller2001SoftwareQuality}, thus providing a tangible contribution to making the migration \gls{business case}. Web migration prototypes are \emph{horizontal prototypes} \autocite{Wallmueller2001SoftwareQuality} focusing on UI and application logic on the client side. Cf. \cref{sec:rwmp}.}
}