\newglossaryentry{artifact} {
  name=Artifact,
  text=artifact,
  plural=artifacts,
  first={\emph{software artifact}}, 
  firstplural={\emph{software artifacts}}, 
  description={``A software artifact is a tangible machine-readable document created during software development. Examples are requirement specification documents, design documents, source code, and executables.'' \autocite{OMG2016KDM}, \autocite[cf.~\emph{physical asset}][entry 3.261]{ISO/IEEE24765Vocabulary}}
}
\newglossaryentry{domain} {
  name=Domain,
  text=domain,
  plural=domains,
  first={\emph{domain}}, 
  firstplural={\emph{domains}}, 
  description={``A sphere of knowledge expressed with a set of concepts and activities that share common aspects that are known to the domain experts. These aspects pertain to the situation, performance, customer interaction, information processing, and more'' \autocite{Chemnitz2017}}
}

\newglossaryentry{domain model} {
  name=Domain Model,
  text=domain model,
  plural=domain models,
  first={\emph{domain model}}, 
  firstplural={\emph{domain models}}, 
  description={``Represents the requirements of the domain by identifying the domain concepts and processes as well as concrete data type, and information and control flow in the domain. This model results from collaboration between web developers and domain experts \autocite{Chemnitz2017}, defined in Definition 7.1.2''}
}


\newglossaryentry{encapsulation} {
  name=Encapsulation,
  text=encapsulation,
  plural=encapsulation,
  first={\emph{encapsulation}}, 
  firstplural={\emph{encapsulation}}, 
  description={``''}
}

\newglossaryentry{ontology} {
  name=Ontology,
  text=ontology,
  plural=ontologies,
  first={\emph{ontology}}, 
  firstplural={\emph{ontologies}}, 
  description={``A logical framework of terms to represent a specific domain of knowledge, encompassing the definitions of the relevant terms and their interrelationships \autocite{Chemnitz2017}, defined in Definition 7.1.3''}
}

\newglossaryentry{Software Component} {
  name=Software Component,
  text=software component,
  plural=Software components,
  first={\emph{software component}}, 
  firstplural={\emph{software components}}, 
  description={``A “self-contained entity, encapsulating operations, and functionalities according to a component model. In other words, components are pieces of data, user interfaces, or application business logic that can be accessed either locally or remotely \autocite{Chemnitz2017}, defined in 4.2.1''}
}

\newglossaryentry{Domain-specific Mashup Tool} {
  name=domain-specific mashup tool,
  text=Domain-specific Mashup Tool,
  plural=domain-specific mashup tools,
  first={\emph{domain-specific mashup tool}}, 
  firstplural={\emph{Domain-specific mashup tools}}, 
  description={``Is designed to meet the domain expert’s requirement  through domain-specific processes manipulating domain concepts \autocite{Chemnitz2017}, defined in Definition 7.1.4''}
}

\newglossaryentry{Concept} {
  name=Concept,
  description={``Concepts are units of human knowledge that can be processed by the human mind (short-term memory) in one instance.'' \autocite{}}
}

\newglossaryentry{conceptual model} {
  name=Conceptual Model,
  text=conceptual model,
    plural=conceptual models,
    first={\emph{conceptual model}}, 
    firstplural={\emph{conceptual models}}, 
  description={``model of the concepts relevant to some endeavor.'' \autocite{Chemnitz2017}}
}


\newglossaryentry{metamodel} {
  name=Metamodel,
  text=metamodel,
  first={\emph{metamodel}}, 
  description={A ``logical information model that specifies the modeling elements used within another (or the same) modeling notation'' \autocite[][entry 3.2433]{ISO/IEEE24765Vocabulary}}
}

\newglossaryentry{Web Engineering} {
  name=Web Engineering,
  first={\emph{Web Engineering}},
  description={``Web Engineering is the application of systematic, disciplined and quantifiable approaches to development, operation, and maintenance of Web-based applications.'' \autocite{Deshpande2002WebEngineering}}
}

