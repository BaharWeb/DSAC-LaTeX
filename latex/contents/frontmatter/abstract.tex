% !TEX root = Clean-Thesis.tex
%
\pdfbookmark[0]{Abstract}{Abstract}
\chapter*{Abstract}
\label{chp:abstract}
%\vspace*{-10mm}

%\lipsum
Rapid technical advances during the last decade have transformed the web into a programmable medium for generating customizable low-cost solutions with limited lifetime as situational applications. Such solutions are used to assist domain experts in the decision-making process within specific environments. However, unique, and ever-changing domain requirements, coupled with the limited technical knowledge possessed by domain experts, hinder precise situational decision-making. Considering the transient purpose of these applications, conventional software development processes have proved to be unprofitable in terms of time and effort. Reflecting the importance of domain specificity, highlighted by several researchers, the lack of standardized domain models and domain-specific customization approaches hinders the full achievement of situational decision-making by domain experts.

This thesis proposed a holistic approach to address the domain specificity of web-based mashups as situational decision-support tools by using scientific approaches. Low expressive power, lack of customizability, and unreasonable usability are identified as the contributing research challenges that need to be tackled to fully resolve the central problem.



%\vspace*{20mm}
%{\usekomafont{chapter}Abstract (different language)}\label{sec:abstract-diff} \\
%\blindtext

\cleardoublepage